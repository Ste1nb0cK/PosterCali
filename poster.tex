\documentclass[final]{beamer}

% ====================
% Packages
% ====================

\usepackage[T1]{fontenc}
\usepackage{lmodern}
\usepackage[orientation=portrait,size=a0,scale=1.0]{beamerposter}
\usetheme{gemini}
\usecolortheme{nott}
\usepackage{graphicx}
\usepackage{braket}
\usepackage{booktabs}
\usepackage{tikz}
\usepackage{pgfplots}
\pgfplotsset{compat=1.14}
\usepackage{anyfontsize}
\usepackage{xcolor}
\usepackage[skip=2pt,font=normalsize]{subcaption}
\usepackage{adjustbox}

% ----------------------------------
% For plotting study methodology
% ----------------------------------

\usepackage{tikz}
\usetikzlibrary{shapes.geometric, arrows}

% Defining Tickz Style
\tikzstyle{startstop} = [rectangle, rounded corners, minimum width=3cm, minimum height=1cm, text centered, text width = 10cm, draw=black, fill=white]

% \tikzstyle{io} = [trapezium, trapezium left angle=70, trapezium right angle=110, minimum width=3cm, minimum height=1cm, text centered, text width = 4.5cm, draw=black, fill=blue!30]

\tikzstyle{process} = [rectangle, minimum width=3cm, minimum height=1cm, text centered, text width = 6cm, draw=black, fill=white, text width = 10cm]

% \tikzstyle{decision} = [diamond, minimum width=3cm, minimum height=1cm, text centered, draw=black, fill=green!30]

\tikzstyle{arrow} = [ultra thick,->,>=stealth]


% ====================
% Lengths
% ====================

% If you have N columns, choose \sepwidth and \colwidth such that
% (N+1)*\sepwidth + N*\colwidth = \paperwidth
\newlength{\sepwidth}
\newlength{\colwidth}
\setlength{\sepwidth}{0.025\paperwidth}
\setlength{\colwidth}{0.45\paperwidth}

\newcommand{\separatorcolumn}{\begin{column}{\sepwidth}\end{column}}

% ====================
% Title
% ====================

\title{Quantum  Super Precision in Parameter Estimation}

\author{Nicolás A. Niño-Salas \inst{1} \and Carlos Viviescas  \inst{2} }

\institute[shortinst]{Universidad Nacional de Colombia, Sede Bogotá }

% ====================
% Footer (optional)
% ====================

\footercontent{
  \href{https://github.com/Ste1nb0cK}{\textbf{https:https://github.com/Ste1nb0cK}} \hfill
  \textbf{Chaos \& Complexity Group} \hfill
  \href{mailto:ninino@unal.edu.co}{\textbf{ninino@unal.edu.co}}}
% (can be left out to remove footer)
% ====================
% Logo (optional)
% ====================

% use this to include logos on the left and/or right side of the header:
\logoright{\includegraphics[height=9cm]{logos/logo_grupo_invertido_fondo_poster.jpg}}
\logoleft{\includegraphics[height=10cm]{logos/logo_u_invertido_poster.jpg}}

% ====================
% Body
% ====================

\begin{document}

\begin{frame}[t]
\begin{columns}[t]
\separatorcolumn

\begin{column}{\colwidth}

% ----------------------------------
% Abstract
% ----------------------------------
  \begin{block}{Abstract}
    The main promise of \textit{\textbf{Quantum Information Technology}} is to surpass all classical limits of information
    processing by embracing the non-classical resources of Quantum
    Theory, and among all the information processing tasks one of the
    most important ones is parameter estimation as most of
    quantitative science requires for appropiate modeling. In this
    work we review the local estimation framework for quantum
    channels  as an appropiate formalism
    for characterizing any physical processes consistent with
    quantum mechanics, then we use it to construct measurement
    independt precision bounds for a Mach-Zender Interferometer,
    identify \textit{\textbf{Squeezed States}} as a non-classical
    aspect giving rise to a quadratic \textit{\textbf{Quantum Advantage}} and finally explore how markovian losses not only
    destroy quantum advantages but lead to the lost of all precision
    in the long encoding limit.
  \end{block}

% ----------------------------------
% Section: Literature review
% ----------------------------------
  \begin{block}{Characterization of Quantum Channels}
  \end{block}
  Characterization of physical processes typically requires the determination of a parameter $\gamma$ in the model used to decribe it e.g.
  a dissipation parameter in a damped harmonic osscillator, its frequency or even the time elapsed. In quantum mechanics
  all possible processes that a systems may undergo can be understood in an \textbf{\textit{Enviroment+System} (SE)} scheme,
%Tweak the space respect to the image
\vspace{-0.025\linewidth}

  % This flow chart is created by the author

% adjustbox is used to limit the figure inside the page
% -- means normal arrow
%  -| horizontal followed by the vertical arrow
%  |- vertical followed by the horizontal arrow
\noident
\hspace{-0.1\linewidth}
\begin{minipage}{0.48\linewidth}
  \centering
\includegraphics[scale=1]{images/SE_Diagram.pdf}
\end{minipage}
\hspace{-0.05\linewidth}
\begin{minipage}{0.48\linewidth}
\includegraphics[scale=1]{images/SE_Diagram_commutation.pdf}
\end{minipage}

%Tweak the space respect to the image
\vspace{-0.2\linewidth}
{\LARGE
  \begin{equation*}
       \varepsilon_{\gamma}\left(\rho_{0}\right)=\mathrm{Tr}_{E}\left[ U_{\gamma}\left( \rho_{0} \otimes \sigma_{0} \right)U_{\gamma}^{\dagger}\right]=\rho_{\gamma}
  \end{equation*}
}
in which the \texttit{\textbf{Open Dynamics}} are induced by the marginalization of the unitary evolution of
$\mathcal{H}_{S}\otimes\mathcal{H}_{E}$, hence the characterization of any physical process can be understood as the estimation of an unknown parameter $\gamma$.
% ----------------------------------
% Section: Research objectives
% ----------------------------------
  \begin{block}{Not All Measurements are Made Equal}
    To actually perform the estimation one needs to measure the system, and for any measurement scheme there exists a \textbf{\textit{Positive
      Operator Valued Measure (POVM)}} $\{\Pi_{x}\}_{x \in \mathbb{R}}$ that describes its statistics:
{\LARGE
  \begin{equation*}
     \wp(x | \gamma)= \mathrm{Tr}\left[\Pi_{x} \rho_{\gamma}\right]
  \end{equation*}
}
from here one sees that once fixed the quantity to be measured our characterization task consists in estimating a parameter from a probability
distribution, hence this is really a problem about \textit{Statistical Inference}, from which we know that any estimation strategy has a variance
bounded from below by the inverse of the \textit{\textbf{Fisher Information (FI)}}:
{\Large
  \begin{align*}
    \Delta^{2}\gamma \geq & \frac{1}{F(\gamma)} & F(\gamma) = \int  \wp(x | \gamma) \left(\frac{\partial}{\partial\gamma} \mathrm{Ln}  [\wp(x |\gamma)]\right)^{2} dx,
  \end{align*}
}
the LHS is known as the \textit{\textbf{Cramer-Rao Bound (CRB)}}. Different POVMs will lead to different FIs, and so in terms precision some
are better than others.
\end{block}

% ----------------------------------
% Section: Study methodology
% ----------------------------------
 \begin{block}{Anatomy of a Metrology Problem}
   The objective of any metrology problem is to estimate the parameter of interest with the minimum possible error, given a fixed amount of a
   certain resource e.g. the mean photon number in the probe.
   % Tweak distance to the image
   \vspace{-0.07\linewidth}
   % This flow chart is created by the author

% adjustbox is used to limit the figure inside the page
% -- means normal arrow
%  -| horizontal followed by the vertical arrow
%  |- vertical followed by the horizontal arrow
\begin{flushright}
\begin{figure}[h]
\includegraphics[scale=1.7]{images/Esquema_estimacion.pdf}
\end{figure}
\end{flushright}

 % Tweak distance to the image
   \vspace{-0.11\linewidth}
   The interesting thing about  \textbf{\textit{Precision Bounds}} is not so much the actual value but the scaling with the number of resources.
   For mean photon numbers we primarily talk about two: \textbf{\textit{Shot-Noise (SN)}} and \textbf{\textit{Heisenberg Scaling (HS)}}:
 {\Large
  \begin{align*}
    \Delta \gamma &\sim \frac{1}{\sqrt{N}} & \Delta \gamma \sim \frac{1}{N}.
  \end{align*}
}
 \end{block}
% -------------------------------
% Section: Site selection and data collection
% -------------------------------
\end{column}
\separatorcolumn

\begin{column}{\colwidth}


\begin{block}{Optimal Protocols and Quantum Metrology}
  It is always possible to saturate the CRB through an appropiate choice of estimator, so the physics problem is to find the maximum
  FI.\ For a fixed initial state this is called the \textit{\textbf{Quantum Fisher Information (QFI)}} $\mathcal{F}(\gamma)$
  and is typically calculated through the \textit{\textbf{Self-Adjoint Logarithmic Derivative (SLD)}}:
  {\Large
  \begin{align*}
    \Lambda \rho_{\gamma} + \Lambda \rho_{\gamma} =& \partial_{\gamma} \rho_{\gamma} & \mathcal{F}(\gamma) = \mathrm{Tr}[\Lambda \partial_{\gamma}\rho_{\gamma}],
  \end{align*}
}
and the POVM that produces it. A few important properties are:
\begin{itemize}
        \item Pure states always offer a higher QFI.
        \item For Unitary Dynamics $U_{\gamma}=\mathrm{exp}({\gamma iG})$, $\mathcal{F}(\gamma)=4\braket{\psi_{0} | G | \psi_{0}}$.
        \item the eigen-projectors of the SLD always define an optimal POVM.
\end{itemize}
The promise of \textbf{\textit{Quantum Metrology}} is to use non-classical behaviors to find \textit{\textbf{Quantum Advantages}} in scaling.
\end{block}


% -------------------------------
% Section: Descriptive Statistics
% -------------------------------

% -------------------------------
% Section: Results and discussion
% -------------------------------

\begin{block}{Examples}
    \heading{Ideal Mach-Zender Interferometer}
    In this configuration we got to arms corresponding to light-modes with a difference frequency $\omega$.
     \vspace{-0.01\linewidth}
     \noident
\hspace{-0.1\linewidth}
\begin{minipage}{0.48\linewidth}
  \centering
\includegraphics[width=\linewidth]{images/MZ_Diagram.pdf}
\end{minipage}
\begin{minipage}{0.5\textwidth}
  \Large{
    \begin{align*}
      V =& \exp\left[ i\theta \left( a_{1}^{\dagger}a_{1} +a_{2}^{\dagger}a_{2}\right) \right]\\
      \\
      U_{\gamma} =& V e^{-iHt}V
    \end{align*}
  }
\end{minipage}
%\hspace{-0.05\linewidth}
%\begin{minipage}{0.48\linewidth}
%\includegraphics[scale=1]{images/SE_Diagram_commutation.pdf}
%\end{minipage}

     \vspace{-0.07\linewidth}
     \heading{Markovian Loss}
     \vspace{-0.02\linewidth}
     To demonstrate the differences with between unitary and non-unitary evolution we look at a light mode with
     losses at zero temperature.
  {\Large
    \begin{equation*}
      \partial_{t}\rho = \frac{\gamma}{2}(2a\rho a^{\dagger}-a^{\dagger}a\rho - \rho a^{\dagger}a)
    \end{equation*}
}
For both cases we are fixing the encoding time $t$.
\end{block}

% -------------------------------
% Section: Conclusions
% -------------------------------
% -------------------------------
% Section: What is already known about this subject?
% -------------------------------
  \begin{exampleblock}{Pseudo-Classical Input: Coherent States}
    \LARGE{
    $\mathcal{F}_{\text{frequency}}(\gamma) = 4t^{2}N\sin^{2}(\theta+\phi)$ }\\
    $\mathcal{F}_{\text{loss}}(\gamma) =& N t^{2}e^{-t\gamma}$
  \end{exampleblock}
% -------------------------------
% Section: What does this study add?
% -------------------------------
  \begin{exampleblock}{Non-Classical Input: Squeezed + Coherent State}
    $$
        \mathcal{F}_{\text{frequency}}(\gamma) &= t^{2}\left(\frac{1-\cos(4\theta)}{4}(N^{2}-M^{2})\left(1+\frac{2}{N-M} \right) +
         N\left(5+\cos(2\theta) - M\left(\frac{1}{2} + \cos(2\theta) \right) \right)\right)
    $$
         \LARGE{$\mathcal{F}_{\text{loss}}(\gamma) = t^{2}N \frac{-2e^{\gamma t} +e^{2\gamma t}+2}{(e^{\gamma t}-1)(2e^{\gamma t}N-2N +e^{2\gamma t})}$}
  \end{exampleblock}
% -------------------------------
% Section: Practical Implications
% -------------------------------
  \begin{exampleblock}{Non-Classical Input: Fock States}

         \LARGE{$\mathcal{F}_{\text{frequency}}(\gamma) = t^{2}\sin(2\theta)\left( N+\frac{N^{2}}{2}-\frac{M^{2}}{2}\right)$}\\
         \LARGE{$\mathcal{F}_{\text{loss}}(\gamma) = t^{2}N \frac{1}{e^{\gamma t}-1}$}
  \end{exampleblock}

% -------------------------------
% Section: References
% -------------------------------

  \begin{block}{References}

    \nocite{*}
    \footnotesize{\bibliographystyle{plain}\bibliography{poster.bib}}

  \end{block}

% -------------------------------
% Section: Portfolio
% -------------------------------
  \begin{block}{Supplemental Material}
    \begin{figure}[h]
    \centering
    \includegraphics[width=0.07\textwidth]{images/portfolio.png}
    \label{fig:figure3}
\end{figure}

  \end{block}

\end{column}
\separatorcolumn
\end{columns}
\end{frame}

\end{document}
