\documentclass[final]{beamer}

% ====================
% Packages
% ====================
 % \usepackage{layouts}
 % \printinunitsof{in}\prntlen{\columnwidth}

\usepackage[T1]{fontenc}
\usepackage{relsize}
\usepackage{lmodern}
\usepackage[orientation=portrait,size=a0,scale=1.11]{beamerposter}
\usetheme{gemini}
\usecolortheme{nott}
\usepackage{graphicx}
\usepackage{braket}
\usepackage{booktabs}
\usepackage{tikz}
\usepackage{pgfplots}
\pgfplotsset{compat=1.14}
\usepackage{anyfontsize}
\usepackage{xcolor}
\usepackage[skip=2pt,font=normalsize]{subcaption}
\usepackage{adjustbox}
\usepackage[backend=biber, style=nature]{biblatex}
\addbibresource{poster.bib}
\renewcommand*{\bibfont}{\footnotesize} % Or \footnotesize, or any other desired size

% ----------------------------------
% For plotting study methodology
% ----------------------------------

\usepackage{tikz}
\usetikzlibrary{shapes.geometric, arrows}

% Defining Tickz Style
\tikzstyle{startstop} = [rectangle, rounded corners, minimum width=3cm, minimum height=1cm, text centered, text width = 10cm, draw=black, fill=white]

% \tikzstyle{io} = [trapezium, trapezium left angle=70, trapezium right angle=110, minimum width=3cm, minimum height=1cm, text centered, text width = 4.5cm, draw=black, fill=blue!30]

\tikzstyle{process} = [rectangle, minimum width=3cm, minimum height=1cm, text centered, text width = 6cm, draw=black, fill=white, text width = 10cm]

% \tikzstyle{decision} = [diamond, minimum width=3cm, minimum height=1cm, text centered, draw=black, fill=green!30]

\tikzstyle{arrow} = [ultra thick,->,>=stealth]


% ====================
% Lengths
% ====================

% If you have N columns, choose \sepwidth and \colwidth such that
% (N+1)*\sepwidth + N*\colwidth = \paperwidth
\newlength{\sepwidth}
\newlength{\colwidth}
\setlength{\sepwidth}{0.025\paperwidth}
\setlength{\colwidth}{0.45\paperwidth}

\newcommand{\separatorcolumn}{\begin{column}{\sepwidth}\end{column}}

% ====================
% Title
% ====================

\title{ Precision Bounds In Bosonic Channels}

\author{Nicolás A. Niño-Salas  \and Carlos Viviescas }

\institute[shortinst]{Universidad Nacional de Colombia, Sede Bogotá }

% ====================
% Footer (optional)
% ====================

\footercontent{
  \href{github.com/Ste1nb0cK}{\textbf{https:https://github.com/Ste1nb0cK}} \hfill
  \textbf{Chaos \& Complexity Group} \hfill
  \href{mailto:ninino@unal.edu.co}{\textbf{ninino@unal.edu.co}}}
% (can be left out to remove footer)
% ====================
% Logo (optional)
% ====================

% use this to include logos on the left and/or right side of the header:
\logoright{\includegraphics[height=9cm]{logos/logo_grupo_invertido_fondo_poster.jpg}}
\logoleft{\includegraphics[height=10cm]{logos/logo_u_invertido_poster.jpg}}

% ====================
% Body
% ====================

\begin{document}

\begin{frame}[t]
\begin{columns}[t]
\separatorcolumn

\begin{column}{\colwidth}

% ----------------------------------
% Abstract
% ----------------------------------
  \begin{block}{Abstract}
    The main promise of \textit{\textbf{Quantum Metrology}} is to surpass all classical limits of \mbox{parameter} estimation
    by embracing the non-classical resources of \textbf{\textit{Quantum
        Systems}}.
    In this work we review the local estimation framework for quantum
    channels, and use it to derive measurement
    independent precision bounds for a Mach-Zender \mbox{Interferometer} and a Markovian Dynamic.
    In the process we identify \textit{\textbf{Squeezed}} and \textit{\textbf{Fock States}} as  non-classical
    states giving rise to quadratic \textit{\textbf{Quantum Advantage}}, and show markovian losses not only
    destroy  the advantage but lead to the loss of all precision in the long encoding limit.

  \end{block}

% ----------------------------------
% Section: Literature review
% ----------------------------------
  \begin{block}{Characterization of Quantum Channels}
  Characterization of physical processes typically requires the determination of a parameter $\gamma$ in the model used to decribe it e.g.
  a dissipation parameter in a damped harmonic osscillator, its frequency or even the time elapsed.%Tweak the space respect to the image
\vspace{-0.04\linewidth}
  % This flow chart is created by the author

% adjustbox is used to limit the figure inside the page
% -- means normal arrow
%  -| horizontal followed by the vertical arrow
%  |- vertical followed by the horizontal arrow
\noident
\hspace{-0.1\linewidth}
\begin{minipage}{0.48\linewidth}
  \centering
\includegraphics[scale=1]{images/SE_Diagram.pdf}
\end{minipage}
\hspace{-0.05\linewidth}
\begin{minipage}{0.48\linewidth}
\includegraphics[scale=1]{images/SE_Diagram_commutation.pdf}
\end{minipage}

%Tweak the space respect to the image
 \vspace{-0.15\linewidth}
{\Large
  \begin{equation*}
       \varepsilon_{\gamma}\left(\rho_{0}\right)=\mathrm{Tr}_{E}\left[ U_{\gamma}\left( \rho_{0} \otimes \sigma_{0} \right)U_{\gamma}^{\dagger}\right]=\rho_{\gamma}
  \end{equation*}
}
  In quantum mechanics all possible processes that a systems may undergo can be viewed in an \textbf{\textit{Enviroment+System} (SE)} scheme,
in which the \textit{\textbf{Open Dynamics}} are induced by the marginalization of the unitary evolution of
$\mathcal{H}_{S}\otimes\mathcal{H}_{E}$: the characterization of any physical process can be understood as the estimation of an unknown parameter $\gamma$.
  \end{block}
% ----------------------------------
% Section: Research objectives
% ----------------------------------
  \begin{block}{Not All Measurements are Made Equal}
    To actually perform the estimation one needs choose a quantity $x$ to measure and a \textbf{\textit{Positive
      Operator Valued Measure (POVM)}} $\{\Pi_{x}\}_{x \in \mathbb{R}}$.
\[\scalebox{1.7}{$\displaystyle \wp(x|\gamma)= \mathrm{Tr}\left[\Pi_{x} \rho_{\gamma}\right] $}\]

The task is estimating a parameter from a probability
distribution: \textit{\textbf{Statistical Inference}}.
Any estimator has a variance bounded from below by the inverse of the \textit{\textbf{Fisher Information (FI)}}:
  \[\scalebox{1.5}{$\displaystyle
\Delta^{2}\gamma\geq\frac{1}{F(\gamma)} \hspace{1cm} F(\gamma) = \int  \wp(x | \gamma) \left(\frac{\partial}{\partial\gamma} \mathrm{Ln}  [\wp(x |\gamma)]\right)^{2} dx.
$}\]
The LHS is known as the \textit{\textbf{Cramer-Rao Bound (CRB)}}.\\
Different POVMs will lead to different FIs: some are better than others.
\end{block}

% ----------------------------------
% Section: Study methodology
% ----------------------------------
 \begin{block}{Anatomy of a Metrology Problem}
   The objective of any metrology problem is, given a fixed amount of resources, to estimate a parameter with the minimum possible error.
   % Tweak distance to the image
   \vspace{-0.01\linewidth}
   % This flow chart is created by the author

% adjustbox is used to limit the figure inside the page
% -- means normal arrow
%  -| horizontal followed by the vertical arrow
%  |- vertical followed by the horizontal arrow
\begin{flushright}
\begin{figure}[h]
\includegraphics[scale=1.7]{images/Esquema_estimacion.pdf}
\end{figure}
\end{flushright}

 % Tweak distance to the image
   \vspace{-0.11\linewidth}
   The interesting thing about \textbf{\textit{Precision Bounds}} is not so much the actual value but how it scales with the number of
   resources.
\[\scalebox{1.3}{$\displaytyle
    \text{\textbf{\textit{Shot-Noise (SN):}}}\hspace{0.5cm}\Delta^{2}\gamma &\sim\frac{1}{N}$}\]
\[\scalebox{1.3}{$\displaytyle
    \text{\textbf{\textit{Heisenberg Scaling (HS):}}}\hspace{0.5cm}\Delta^{2}\gamma &\sim\frac{1}{N^{2}}$}\]

 \end{block}
% -------------------------------
% Section: Site selection and data collection
% -------------------------------
 \begin{block}{Optimal Protocols and Quantum Metrology}
  The CRB can be saturated by an appropiate choice of estimator. For a fixed initial state $\rho_{0}$ we maximize the
  FI, called the \textit{\textbf{Quantum Fisher Information (QFI)}} $\mathcal{F}(\gamma)$
   and usually calculated through the \textit{\textbf{Self-Adjoint Logarithmic Derivative (SLD)}}:
   \\
   \[\scalebox{1.3}{$\displaystyle
     \mathcal{F}(\gamma) = \mathrm{Tr}[\Lambda \partial_{\gamma}\rho_{\gamma}] \hspace{5cm}  \Lambda \rho_{\gamma} +  \rho_{\gamma} \Lambda= 2\partial_{\gamma} \rho_{\gamma}
     $}\]

  Properties:
\begin{itemize}
        \item Pure states always offer a higher QFI.
        \item For Unitary Dynamics $U_{\gamma}=\exp({\gamma iG})$, $\mathcal{F}(\gamma)=4\Delta^{2}G$.
        \item the eigen-projectors of the SLD always define an optimal POVM.
\end{itemize}
\end{block}
\end{column}
\separatorcolumn
\begin{column}{\colwidth}

% -------------------------------
% Section: Descriptive Statistics
% -------------------------------

% -------------------------------
% Section: Results and discussion
% -------------------------------
\begin{block}{Unitary vs Markovian Non-unitary}
    \heading{Ideal Mach-Zender Interferometer}
     \noident
\hspace{-0.1\linewidth}
\begin{minipage}{0.48\linewidth}
  \centering
\includegraphics[width=\linewidth]{images/MZ_Diagram.pdf}
\end{minipage}
\begin{minipage}{0.5\textwidth}
  \Large{
    \begin{align*}
      V =& \exp\left[ i\theta \left( a_{1}^{\dagger}a_{1} +a_{2}^{\dagger}a_{2}\right) \right]\\
      \\
      U_{\gamma} =& V e^{-iHt}V
    \end{align*}
  }
\end{minipage}
%\hspace{-0.05\linewidth}
%\begin{minipage}{0.48\linewidth}
%\includegraphics[scale=1]{images/SE_Diagram_commutation.pdf}
%\end{minipage}

     % \Large{
     % \begin{align}
       % V =& e^{i\theta\left(  a_{2}a_{1}^{\dagger}+a_{1}a_{2}^{\dagger}\right)} &
                                                                                  % U = Ve^{-iHt}V
     % \end{align}}
     \heading{Zero Temperature Enviroment}
     \vspace{-0.05\linewidth}
  {\Large
    \begin{equation*}
      \partial_{t}\rho = \frac{\gamma}{2}(2a\rho a^{\dagger}-a^{\dagger}a\rho - \rho a^{\dagger}a)
    \end{equation*}
}
Fixing the encoding time to $t$.
\end{block}

% -------------------------------
% Section: Conclusions
% -------------------------------
% -------------------------------
% Section: What is already known about this subject?
% -------------------------------
  \begin{exampleblock}{Results}
    The QFI for big mean photon number $N$ is presented below for various probes and  the minimum relative variance for the loss estimation as a function of $\gamma t$.
    \begin{table}
      \resizebox{0.8\columnwidth}{!}{
      \begin{tabular}{c c c c }
        \toprule
        &\textbf{Coherent} & \textbf{Squeezed}  & \textbf{Fock}   \\
        \midrule
        Frequency & $Nt^2$& $t^{2} (\frac{1}{2}N^{2}+5N )$ &  $t^{2}(\frac{N^{2}}{2}+N)$\\
        Loss & $N t^{2}e^{-t\gamma}$ & $ t^{2}N \frac{-2e^{\gamma t} +e^{2\gamma t}+2}{(e^{\gamma t}-1)(2e^{\gamma t}N-2N +e^{2\gamma t})}$ & $t^{2}N \frac{1}{e^{\gamma t}-1}$\\
        \bottomrule
      \end{tabular}}
      \label{tab:table1}
\end{table}

    \vspace{-0.1\linewidth}
    \begin{figure}[h]
    \centering
    \includegraphics[width=0.07\textwidth]{images/portfolio.png}
    \label{fig:figure3}
\end{figure}

    \vspace{-0.1\linewidth}
  \end{exampleblock}
\begin{alertblock}{Discussion}
  \begin{itemize}
    \item In the unitary case \textbf{Non-Classical Probes} have a quadratic advantage in the photon number.
    \item The precision bound for the unitary case is independent on the actual value of the parameter.
    \item Contrary to the scaling with the number of probes, in photon mean number  \textbf{Super HS} scaling is possible \cite{jiao_quantum_2023}.
    \item Markovian noise degrades the scalings back to \textbf{SN}, and makes the \textbf{QFI} decay exponentially.
    \item In the loss estimation there is an optimal encoding time.
  \end{itemize}
\end{alertblock}
\begin{alertblock}{Conclusions and Future Work}
  \begin{itemize}
    \item Quantum systems offer a considerable advantage in parameter estimation.
    \item Quantum Information Theory gives fundamental precision bounds.
    \item Markovian noise constitutes a no-go in practical applications
    \item Non-markovian models and control protocols are necessary for \textbf{Q-Metrology} to be a viable technology.
  \end{itemize}
\end{alertblock}
% -------------------------------
% Section: References
% -------------------------------
\begin{block}{Acknowledgement}
  The authors worked in this project as part of the collaboration \textit{Proyecto de regalías: Tecnologías cuánticas útiles para metrología y Computación Cuántica.}
  \end{block}
  \begin{block}{References}

\nocite{*}
    \small
    {
    \printbibliography{}}
  %
  \end{block}


\end{column}
\separatorcolumn
\end{columns}
\end{frame}

\end{document}
