\documentclass[final]{beamer}

% ====================
% Packages
% ====================

\usepackage[T1]{fontenc}
\usepackage{lmodern}
\usepackage[orientation=portrait,size=a0,scale=1.0]{beamerposter}
\usetheme{gemini}
\usecolortheme{nott}
\usepackage{graphicx}
\usepackage{booktabs}
\usepackage{tikz}
\usepackage{pgfplots}
\pgfplotsset{compat=1.14}
\usepackage{anyfontsize}
\usepackage{xcolor}
\usepackage[skip=2pt,font=normalsize]{subcaption}
\usepackage{adjustbox}

% ----------------------------------
% For plotting study methodology
% ----------------------------------

\usepackage{tikz}
\usetikzlibrary{shapes.geometric, arrows}

% Defining Tickz Style
\tikzstyle{startstop} = [rectangle, rounded corners, minimum width=3cm, minimum height=1cm, text centered, text width = 10cm, draw=black, fill=white]

% \tikzstyle{io} = [trapezium, trapezium left angle=70, trapezium right angle=110, minimum width=3cm, minimum height=1cm, text centered, text width = 4.5cm, draw=black, fill=blue!30]

\tikzstyle{process} = [rectangle, minimum width=3cm, minimum height=1cm, text centered, text width = 6cm, draw=black, fill=white, text width = 10cm]

% \tikzstyle{decision} = [diamond, minimum width=3cm, minimum height=1cm, text centered, draw=black, fill=green!30]

\tikzstyle{arrow} = [ultra thick,->,>=stealth]


% ====================
% Lengths
% ====================

% If you have N columns, choose \sepwidth and \colwidth such that
% (N+1)*\sepwidth + N*\colwidth = \paperwidth
\newlength{\sepwidth}
\newlength{\colwidth}
\setlength{\sepwidth}{0.025\paperwidth}
\setlength{\colwidth}{0.45\paperwidth}

\newcommand{\separatorcolumn}{\begin{column}{\sepwidth}\end{column}}

% ====================
% Title
% ====================

\title{Precision Bounds in Bosonic Channels}

\author{Nicolás A. Niño-Salas \inst{1} \and Carlos Viviescas  \inst{2} }

\institute[shortinst]{Universidad Nacional de Colombia, Sede Bogotá }

% ====================
% Footer (optional)
% ====================

\footercontent{
  \href{https:https://github.com/Ste1nb0cK}{\textbf{https:https://github.com/Ste1nb0cK}} \hfill
  \textbf{Chaos \& Complexity Group} \hfill
  \href{mailto:ninino@unal.edu.co}{\textbf{ninino@unal.edu.co}}}
% (can be left out to remove footer)
% ====================
% Logo (optional)
% ====================

% use this to include logos on the left and/or right side of the header:
\logoright{\includegraphics[height=9cm]{logos/logo_grupo_invertido_fondo_poster.jpg}}
\logoleft{\includegraphics[height=10cm]{logos/logo_u_invertido_poster.jpg}}

% ====================
% Body
% ====================

\begin{document}

\begin{frame}[t]
\begin{columns}[t]
\separatorcolumn

\begin{column}{\colwidth}

% ----------------------------------
% Abstract
% ----------------------------------
  \begin{block}{Abstract}
  \end{block}

% ----------------------------------
% Section: Literature review
% ----------------------------------
  \begin{block}{1.1 Introduction}
  \end{block}
  Characterization of physical processes typically requires the determination of a parameter $\gamma$ in the model used to decribe it e.g.
  a dissipation parameter in a damped harmonic osscillator, its frequency or even the time elapsed. In quantum mechanics
  all possible processes that a systems can undergo are described by a \textit{Completly Positive Trace Preserving} (\textbf{CPTP}) map, also
  called a quantum channel, and all can be realized by:
{\LARGE
  \begin{equation*}
       \varepsilon_{\gamma}\left(\rho_{0}\right)=\mathrm{Tr}_{E}\left[ U_{\gamma}\left( \rho_{0} \otimes \sigma_{0} \right)U_{\gamma}^{\dagger}\right]=\rho_{\gamma}
  \end{equation*}
}

  hence the characterization of any physical process can be understood as the estimation of an unknown parameter $\gamma$.
% ----------------------------------
% Section: Research objectives
% ----------------------------------
  \begin{block}{1.2 Measurement does Matter After All}
    To actually perform the estimation one needs to measure the system, and for any measurement scheme there exists a \textit{Positive
      Operator Values Measure} (\textbf{POVM}) $\{\Pi_{x}\}_{x \in \mathbb{R}}$ that describes its statistics:
{\LARGE
  \begin{equation*}
     \wp(x | \gamma)= \mathrm{Tr}\left[\Pi_{x} \rho_{\gamma}\right]
  \end{equation*}
}
from here one sees that once fixed the quantity to be measured our characterization task consists in estimating a parameter from a probability
distribution, hence this is really a problem about \textit{Statistical Inference}.




  \end{block}

% ----------------------------------
% Section: Study methodology
% ----------------------------------
 \begin{block}{1.3 Anatomy of a Metrology Protocol}
   % This flow chart is created by the author

% adjustbox is used to limit the figure inside the page
% -- means normal arrow
%  -| horizontal followed by the vertical arrow
%  |- vertical followed by the horizontal arrow
\begin{flushright}
\begin{figure}[h]
\includegraphics[scale=1.7]{images/Esquema_estimacion.pdf}
\end{figure}
\end{flushright}

 \end{block}

% -------------------------------
% Section: Site selection and data collection
% -------------------------------
\begin{block}{1.4 Optimal Protocols}

\end{block}
\begin{block}{1.5 Unitary Case}

\end{block}


% -------------------------------
% Section: Descriptive Statistics
% -------------------------------
\end{column}

\separatorcolumn

\begin{column}{\colwidth}

% -------------------------------
% Section: Results and discussion
% -------------------------------

\begin{block}{Examples}
    \heading{Ideal Mach Zender Interferometer}

    % \noident
\hspace{-0.1\linewidth}
\begin{minipage}{0.48\linewidth}
  \centering
\includegraphics[width=\linewidth]{images/MZ_Diagram.pdf}
\end{minipage}
\begin{minipage}{0.5\textwidth}
  \Large{
    \begin{align*}
      V =& \exp\left[ i\theta \left( a_{1}^{\dagger}a_{1} +a_{2}^{\dagger}a_{2}\right) \right]\\
      \\
      U_{\gamma} =& V e^{-iHt}V
    \end{align*}
  }
\end{minipage}
%\hspace{-0.05\linewidth}
%\begin{minipage}{0.48\linewidth}
%\includegraphics[scale=1]{images/SE_Diagram_commutation.pdf}
%\end{minipage}


    % \begin{table}
      \resizebox{0.8\columnwidth}{!}{
      \begin{tabular}{c c c c }
        \toprule
        &\textbf{Coherent} & \textbf{Squeezed}  & \textbf{Fock}   \\
        \midrule
        Frequency & $Nt^2$& $t^{2} (\frac{1}{2}N^{2}+5N )$ &  $t^{2}(\frac{N^{2}}{2}+N)$\\
        Loss & $N t^{2}e^{-t\gamma}$ & $ t^{2}N \frac{-2e^{\gamma t} +e^{2\gamma t}+2}{(e^{\gamma t}-1)(2e^{\gamma t}N-2N +e^{2\gamma t})}$ & $t^{2}N \frac{1}{e^{\gamma t}-1}$\\
        \bottomrule
      \end{tabular}}
      \label{tab:table1}
\end{table}


\end{block}

% -------------------------------
% Section: Conclusions
% -------------------------------
   \begin{block}{Conclusions}
     Todas las veo buenassssss, si bebo ronnnnnnnnnnnnn
  \end{block}


% -------------------------------
% Section: What is already known about this subject?
% -------------------------------
  \begin{exampleblock}{Pseudo-Classical Input: Coherent States}
MUJERIEGOOOOOOOOOOOOOOOO

  \end{exampleblock}


% -------------------------------
% Section: What does this study add?
% -------------------------------
  \begin{exampleblock}{Non-Classical Input: Squeezed + Coherent States}

  \end{exampleblock}


% -------------------------------
% Section: Practical Implications
% -------------------------------
  \begin{exampleblock}{Non-Classical Input: Fock States}
POR LA CALLE EL DINERO Y EL ACOHOL
  \end{exampleblock}

% -------------------------------
% Section: References
% -------------------------------

  \begin{block}{References}

    \nocite{*}
    \footnotesize{\bibliographystyle{plain}\bibliography{poster.bib}}

  \end{block}

% -------------------------------
% Section: Portfolio
% -------------------------------
  \begin{block}{Supplemental Material}

    \begin{figure}[h]
    \centering
    \includegraphics[width=0.07\textwidth]{images/portfolio.png}
    \label{fig:figure3}
\end{figure}


  \end{block}

\end{column}
\separatorcolumn
\end{columns}
\end{frame}

\end{document}
